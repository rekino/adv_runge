\documentclass[12pt,onecolumn,a4paper]{article}
\usepackage{epsfig,graphicx,subfigure,amsthm,amsmath}
\usepackage{color,xcolor}
\usepackage{pgfplots}
\pgfplotsset{compat=newest}
\usepackage{tikz}
\usepackage{xepersian}
\settextfont[Scale=1.2]{XB Zar}
\setlatintextfont[Scale=1]{Times New Roman}

\begin{document}
\title{توجیه وجود مثال‌های خصمانه \\ و انتقال‌پذیری آن‌ها} 
\author{رامین براتی}
\date{\today}
\maketitle

\begin{abstract}
چکیده
\end{abstract}

\section{مقدمه} 
مقدمه.

\section{کارهای مرتبط}
دیدگاه پاکت‌های نادر.
دیدگاه خطی بودن.
دیدگاه کج کردن مرز.

\section{پدیده‌ی رونگه و مثال‌های خصمانه}
در این بخش، دیدگاه جدیدی را در مورد دلیل وجود مثال‌های خصمانه و انتقال‌پذیری آن‌ها ارائه می‌دهیم. در این دیدگاه، مثال‌های خصمانه را اثر جانبی پدیده‌ی رونگه دانسته و انتقال‌پذیری آن‌ها را با استفاده از اصل داربو توجیه می‌کنیم. ابتدا پدیده‌ی رونگه را توصیف می‌کنیم. در آنالیز عددی، پدیده‌ی رونگه مشکلی است که هنگام درون‌یابی با استفاده از چندجمله‌ای‌های درجه بالا به وجود می‌آید. طبق قضیه‌ی تقریب وایرشتراس
\LTRfootnote{Weierstrass approximation theorem}،
هر تابع پیوسته‌ی 
$f(x)$
که بر روی یک بازه‌ی
$[a,b]$
تعریف شود را می‌توان با استفاده از یک چندجمله‌ای درون‌یاب
$P_n(x)$
تقریب زد. به عبارت دقیق‌تر
\begin{equation*}
    \lim_{n\rightarrow \infty}\left(\max _{{a\leq x\leq b}}\left|f(x)-P_{n}(x)\right|\right)=0.
\end{equation*}
طبق این قضیه، طبیعی است که انتظار داشته باشیم که با افزایش 
$n$
به تقریب دقیق‌تری از تابع مورد نظر دست یابیم. با این وجود، ممکن است که چندجمله‌ای‌هایی که انتخاب شده‌اند خاصیت همگرایی یکپارچه را نداشته باشند. توجه کنید که قضیه‌ی تقریب وایرشتراس فقط وجود این چندجمله‌ای را تضمین می‌کند، و مسیری برای یافتن این چندجمله‌ای معرفی نمی‌کند. در واقع، چندجمله‌ای‌هایی که به این صورت ساخته می‌شود ممکن است با افزایش درجه‌ی چندجمله‌ای واگرا شوند. این مشکل به صورت الگوهای نوسانی بروز کرده که در نزدیکی نقاط درون‌یابی انتهایی بازه افزایش می‌یابد. شناسایی این پدیده را به  کارل رونگه
\LTRfootnote{Carl David Tolme Runge}،
ریاضیدان آلمانی، منتصب می‌کنند.

تابع رونگه با تعریف 
$f(x)=\frac{1}{1+25x^2}$ 
را در نظر بگیرید. چندجمله‌ای درون‌یاب این تابع را با استفاده از نقاط 
$x_i$ 
تعریف می‌کنیم، به طوری که
$x_{i}=\frac{2i}{n}-1$ 
و 
$i\in \left\{0,1,\dots ,n\right\}$.
همان گونه که مشاهده می‌شود، نقاط 
$x_i$ 
در بازه‌ی 
$[1,-1]$ 
و با فاصله‌ی یکسان توزیع شده‌اند. رونگه متوجه شد که چندجمله‌ای درون‌یابی که به روش ذکر شده ساخته شود، در نزدیکی مرز‌های دامنه‌ی تابع نوسان می‌کند. حتی می‌توان نشان داد که خطای درون‌یابی با افزایش 
$n$ 
 به صورت بیکران افرایش می‌یابد.

 \begin{figure}
     \begin{tikzpicture}
        \begin{axis}[xmin=-1,xmax=1,xmin=-1.2,xmax=1.2,samples=100]
            \addplot[domain=-1:1,blue,dashed](x,1/(1+25*x*x));
        \end{axis}
     \end{tikzpicture}
 \end{figure}

\section{نتایج}
نتایج.

\end{document}